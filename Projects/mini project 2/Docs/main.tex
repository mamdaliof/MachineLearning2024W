\documentclass{article}

\usepackage{graphicx}
\usepackage{fancyhdr}
\usepackage[margin=1in]{geometry}
\usepackage{listings}
\usepackage[hidelinks]{hyperref}
%\usepackage{subfigure}
\usepackage{subcaption}

\usepackage{amsmath}
\usepackage{subcaption}
\usepackage{float}
\usepackage{booktabs}

\hypersetup{
	colorlinks=true,
	linkcolor=teal,
	filecolor=magenta,      
	urlcolor=teal,
	citecolor = teal
}
\usepackage{xcolor}
\usepackage{xepersian}
\setlength\headheight{28pt} 
\settextfont[
    Path = ./font/, % specify the path to the font files
    Scale = 1.3,
    UprightFont = XB Niloofar, % Regular font
    BoldFont = XB NiloofarBd,       % Bold font
    ItalicFont = XB NiloofarIt,   % Italic font
    BoldItalicFont = XB NiloofarBdIt % Bold Italic font if available
]{XB Niloofar}
\setlatintextfont[Scale=1]{Times New Roman}
\renewcommand{\baselinestretch}{1.5}
\pagestyle{fancy}
\fancyhf{}
\rhead{\includegraphics[width=1cm]{img/Logo.png} 
	یادگیری ماشین
	-
	مینی پروژه شماره 2}
\lhead{\thepage}
\rfoot{سیدمحمد حسینی}
\lfoot{9821253}
\renewcommand{\headrulewidth}{1pt}
\renewcommand{\footrulewidth}{1pt}
\renewcommand{\lstlistingname}{Code}

\definecolor{codegreen}{rgb}{0,0.6,0}
\definecolor{codegray}{rgb}{0.5,0.5,0.5}
\definecolor{codepurple}{rgb}{0.58,0,0.82}
\definecolor{backcolour}{rgb}{0.95,0.95,0.92}

\lstdefinestyle{mystyle}{
	backgroundcolor=\color{backcolour},   
	commentstyle=\color{codegreen},
	keywordstyle=\color{magenta},
	numberstyle=\tiny\color{codegray},
	stringstyle=\color{codepurple},
	basicstyle=\ttfamily\footnotesize,
	breakatwhitespace=false,         
	breaklines=true,                 
	captionpos=b,                    
	keepspaces=true,                 
	numbers=left,                    
	numbersep=5pt,                  
	showspaces=false,                
	showstringspaces=false,
	showtabs=false,                  
	tabsize=2
}

\lstset{style=mystyle}

\begin{document}
	
	\begin{titlepage}
\begin{center}
\defpersianfont\nast[Path={./font/}, Scale=2]{IranNastaliq}
\centerline{{\includegraphics[width=5cm]{img/Logo.png}}}
\centerline{\textcolor[rgb]{0,0,0.5}{\nast \large  دانشگاه صنعتی خواجه نصیرالدین طوسی}}
\centerline{\textcolor[rgb]{0,0,0.5}{\nast \bfseries دانشکدۀ مهندسی برق - گروه مهندسی کنترل}}

\vfill
        
\Huge
\textbf{یادگیری ماشین}\\
\textbf{پاسخ مینی پروژه دوم}\\
        
\vfill
\large
    نام و نام خانوادگی:  سیدمحمد حسینی\\
    شمارۀ دانشجویی: 9901399\\
    تاریخ: مهرماه 1402\\

\href{https://github.com/mamdaliof/MachineLearning2024W}{GitHub}

\href{https://drive.google.com/drive/folders/10-8uiqpIUbC2HcEtwjqP1IAix46nCrbL?usp=sharing}{Google Drive}


\end{center}
\end{titlepage}

	
	\tableofcontents \clearpage
	\listoffigures \clearpage
	\listoftables \clearpage
	\lstlistoflistings \clearpage
	\newpage
	
	\section{سوال 1}\label{Section1}
	\subsection{بخش اول}
	{\large \textbf{سوال}:} \\
	فرض کنید در یک مسألۀ طبقه بندی دوکلاسه، دو لایۀ انتهایی شبکۀ شما فعال ساز 
	\lr{ReLU}
	و سیگموید است. چه اتفاقی می افتد؟‎\\
	
	{\large \textbf{پاسخ}:} 
	
	توابع فعال‌ساز
	\lr{Relu}
	و
	\lr{Sigmoid}
	جزو پرکاربردترین اجزاء در یک مدل یادگیری ماشین هستند که به‌منظور اضافه‌کردن خاصیت غیرخطی به مدل مورداستفاده قرار می‌گیرند. به‌منظور تحلیل اثر این دو تابع فعال‌ساز بهتر است ابتدا خواص هرکدام را به‌صورت مجزا بررسی کنیم.
	\begin{itemize}
		\item تابع فعال‌سازی
		\lr{Sigmoid}:\\
		تابع سیگموید مقداری بین 0 و 1 خروجی می‌دهد که می‌تواند به‌عنوان احتمال تفسیر شود. این تابع فعال‌سازی اصولاً به‌عنوان فعال‌ساز در آخرین لایه مدل‌های طبقه‌بندی استفاده می‌شود؛ زیرا تفسیر احتمالی مستقیمی از خروجی ارائه می‌دهد. از دیگر خواص تابع سیگموید مشتق‌پذیری آن است که از لحاظ برنامه‌نویسی کار را بسیار ساده می‌کند؛ زیرا همان‌طور که در
		\autoref{sigmoid_d}
		مشاهده می‌کنید مشتق تابع سیگموید رابطه‌ای است متشکل از خود تابع سیگموید.
		\begin{equation}
			\label{sigmoid_d}
			\frac{d}{dx}\sigma(x) = \sigma(x) \cdot (1 - \sigma(x))
		\end{equation}
		\item تابع فعال‌سازی
		\lr{ReLU}:\\
		تابع
		\lr{ReLU}
		ساده‌ترین تابع فعال‌سازی غیرخطی است که مورداستفاده قرار می‌گیرد، این تابع به‌ازای مقادیر مثبت تابع همانی است و به‌ازای مقادیر منفی عدد 0 را به‌عنوان خروجی در نظر می‌گیرد که این مسئله به‌سادگی محاسبات هنگام
		\lr{Back propagation}
		بسیار کمک می‌کند. مشتق این تابع به‌ازای مقادیر مثبت برابر 1 و به‌ازای مقادیر منفی برابر 0 است.
	\end{itemize}
	حال باتوجه‌به توضیحات فوق می‌توانیم یک شبکه که دولایه انتهایی آن متشکل از یک تابع
	\lr{ReLU} و \lr{Sigmoid}
	است را بررسی کنیم. در این شبکه اطلاعات به‌دست‌آمده از
	\lr{backbone}
	شبکه به لایه یکی مانده به آخر ارسال می‌شود که با انجام یک عملیات خطی این اطلاعات به محیط دیگری تصویر می‌شوند که می‌توانند مقادیر مثبت و منفی را با هر اندازه‌ای داشته باشند، این مسئله به مقادیر موجود در
	\lr{Wight} و \lr{Bias}
	نورون بستگی دارد. حال خروجی نرون‌ها به لایه فعال‌ساز ارسال می‌شوند که در نتیجه آن مقادیر مثبت تبدیل خطی نگه داشته می‌شوند و مقادیر منفی به عدد 0 تصویر می‌شوند. باید توجه داشت که فعال‌ساز
	\lr{ReLU}
	به تعبیری همانند یک ماژول
	\lr{Attention}
	عمل می‌کند و در فرایند آموزش شبکه
	\lr{Wight} و \lr{Bias}
	نرون‌ها را به سمتی متمایل می‌کند که در صورت پیدانکردن
	\lr{Feature}
	معنی‌دار از ورودی خود، یک عدد منفی تولید کند. ایراد فعال‌ساز
	\lr{ReLU}
	این است که اگر لایه‌های پشت‌هم در شبکه، همگی مقدار مثبت تولید کنند، تمام تبدیل‌های خطی موجود در این لایه‌ها می‌توانند با یک تبدیل خطی شبیه‌سازی شوند که این مسئله باعث کاهش پیچیدگی مدل ما خواهد شد؛ بنابراین استفاده از روش‌های رگولاریزیشن و
	\lr{Batch Normalization}
	در کنار توابع فعال‌سازی که به‌صورت پیوسته رفتار غیرخطی دارند موجب رفع این مشکل خواهد شد. در ادامه این مسئله خروجی لایه یکی مانده به آخر وارد لایه تصمیم‌گیری خواهد شد؛ ورودی این لایه تماماً اعداد مثبت است و تبدیل خطی برای اینکه بتواند کلاس مثبت را از منفی جدا کند باید به‌گونه‌ای تنظیم شود که بتواند ورودی‌های تماماً مثبت را به یک عدد مثبت یا منفی تبدیل کند؛ زیرا تابع سیگموید در نقطه 0 مقدار 0.5 را دارد و اگر تبدیل خطی نتواند اعداد مثبت و منفی را از ورودی‌های تماماً مثبت استخراج کند، تابع سیگموید توانایی ایجاد خروجی مناسب را ندارد.
	
	\subsection{بخش دوم}
	{\large \textbf{سوال}:} 
	
	یک جایگزین برای 
	\lr{ReLU}
	تابع
	\lr{ELU}
	می‌باشد . ضمن محاسبه گرادیان آن، حداقل یک مزیت آن را مطرح کنید
	
	{\large \textbf{پاسخ}:} 
	\\ابتدا مشتق این تابع را محاسبه می‌کنیم:
	\begin{align}
		\label{ELU}
		ELU(x) &=
		\begin{cases}
			x & \text{if } x > 0 \\
			\alpha(e^x - 1) & \text{if } x \leq 0
		\end{cases} 
		\Rightarrow \nonumber \\
		\frac{d}{dx} ELU(x) &=
		\begin{cases}
			1 & \text{if } x > 0 \\
			\alpha e^x  & \text{if } x \leq 0
		\end{cases}
	\end{align}
	
	همانطور که در 
	\autoref{ELU}
	مشخص است مشتق این تابع برای مقادیر مثبت با تابع
	\lr{ReLU}
	تفاوتی ندارد اما برای مقادیر منفی برابر 
	$e^x$
	می‌باشد. همانطور که در قسمت قبل مطرح شد تابع
	\lr{ReLU}
	می‌تواند موجب این شود که در عمل تعداد لایه‌های شبکه کاهش پیدا کند اما با استفاده از تابع فعال‌ساز
	\lr{ELU}
	و استفاده از تکنیک‌های مناسب، می‌توانیم به این مشکل غلبه کنیم. علاوه‌بر این تابع
	\lr{ELU}
	در مشتق خود ناپیوستگی ندارد و  بر خلاف 
	\lr{ReLU}‎
	مقدار آن به ازای مقادیر منفی به آرامی برابر 
	$-\alpha$
	می‌شود.‎\cite{ml-cheatsheet-activation-functions}‎ 
	
	\subsection{بخش سوم}
	در ابتدا با استفاده از توزیع یکنواخت تعداد 2000 داده ایجاد می‌کنیم و برچسب نقاطی که داخل مثلث مورد نظر هستند را به مقدار 1 و برچسب بقیه نقاط را 0 می‌کنیم. این عملات توسط کد زیر انجام گرفته است.
	\begin{LTR}
	\begin{lstlisting}[language=Python, caption=data generation]

	def point_in_triangle(point, v1, v2, v3):
		"""Check if point (px, py) is inside the triangle with vertices v1, v2, v3."""
		# Unpack vertices
		x1, y1 = v1
		x2, y2 = v2
		x3, y3 = v3
		px, py = point
		
		# Vectors
		v0 = (x3 - x1, y3 - y1)
		v1 = (x2 - x1, y2 - y1)
		v2 = (px - x1, py - y1)
		
		# Dot products
		dot00 = np.dot(v0, v0)
		dot01 = np.dot(v0, v1)
		dot02 = np.dot(v0, v2)
		dot11 = np.dot(v1, v1)
		dot12 = np.dot(v1, v2)
		
		# Barycentric coordinates
		invDenom = 1 / (dot00 * dot11 - dot01 * dot01)
		u = (dot11 * dot02 - dot01 * dot12) * invDenom
		v = (dot00 * dot12 - dot01 * dot02) * invDenom
		
		# Check if point is in triangle
		return (u >= 0) and (v >= 0) and (u + v < 1)
		
	# Triangle vertices
	v1 = (1, 0)
	v2 = (2, 2)
	v3 = (3, 0)
		
	# Generate random points
	np.random.seed(53)
		
	x_coords = np.random.uniform(0, 4, 2000)
	y_coords = np.random.uniform(-1, 3, 2000)
	x_train = np.column_stack((x_coords, y_coords))
		
	x_coords = np.random.uniform(0, 4, 500)
	y_coords = np.random.uniform(-1, 3, 500)
	x_test = np.column_stack((x_coords, y_coords))
		
	# Label points based on whether they are inside the triangle
	y_train = np.array([point_in_triangle(pt, v1, v2, v3) for pt in x_train]).astype(int)
	y_test = np.array([point_in_triangle(pt, v1, v2, v3) for pt in x_test]).astype(int)

	\end{lstlisting}
	\end{LTR}
کد فوق به منظور ایجاد 2000 نقطه و بررسی اینکه هر نقطه درون مختصات مثلث قرار دارد یا خیر نوشته شده است. تابع
 \texttt{point\_in\_triangle}
  بررسی می‌کند که آیا یک نقطه داخل مثلثی با رئوس داده شده قرار دارد. مختصات رئوس مثلث و نقطه مورد نظر به تابع داده می‌شودو سپس نسبت مختصات نقطه به مثلث مشخص می‌شود. نقاط تصادفی برای مجموعه‌های آموزشی و تست تولید و با استفاده از تابع مذکور برچسب‌گذاری می‌شوند. همانطور که در 
  \autoref{fig: Q1 data}
  نمایش داده شده است، دو مجموعه داده به منظور آموزش و ارزیابی مدل تولید شده است.
  


\begin{figure}[h] 
	\centering
	\subfigure[]{
		\includegraphics[width=0.4\linewidth]{img/Q1_train_set.png}
		}
	\subfigure[]{
		\includegraphics[width=0.4\linewidth]{img/Q1_test_set.png}
		}
	\caption{مجموعه داده تولید شده}
	\label{fig: Q1 data}
\end{figure}

در ادامه یک مدل ‎\lr{MLP}‎‎ روی این مجموعه داده آموزش دیده است و نتایج آن به همراه 
\lr{Decision boundary}‎ 
آن نمایش داده شده است. مدل اول \lr{MLP}‎ با استفاده از توابع \lr{ReLU}‎ ایجاد شده است که جزییات آن به شرح زیر است:
	
\begin{LTR}
\begin{verbatim}
	MLP(
	(layers): Sequential(
	(0): Linear(in_features=2, out_features=8, bias=True)
	(1): ReLU()
	(2): Linear(in_features=8, out_features=64, bias=True)
	(3): ReLU()
	(4): Linear(in_features=64, out_features=8, bias=True)
	(5): ReLU()
	(6): Linear(in_features=8, out_features=1, bias=True)
	(7): Sigmoid()
	)
	)
\end{verbatim}
\end{LTR}
در ادامه مدل فوق با 
\lr{config}
زیر به میزان 150 
\lr{Epoch}
آموزش داده شده است.
\begin{LTR}
	\begin{lstlisting}[language=Python, caption= Configuration]
		device = "cuda" if torch.cuda.is_available else "cpu"
		model = MLP(input_size=2, hidden_size1=8, hidden_size2=64, hidden_size3=8, output_size=1).to(device)
		optimizer = optim.Adam(model.parameters(), lr=0.001)		
		criterion = nn.BCELoss()  # For binary classification
		# DataLoader
		train_loader = DataLoader(TensorDataset(x_train_tensor, y_train_tensor), batch_size=128, shuffle=True)
		test_loader = DataLoader(TensorDataset(x_test_tensor, y_test_tensor), batch_size=512, shuffle=False)
				
	
	\end{lstlisting}
\end{LTR}
کد زیر به منظور اجرای حلقه آموزش نوشته شده است.
در این کد، چندین متغیر برای نگه‌داری اطلاعات مانند تاریخچه‌ی خطاها و معیارهای متریکی مورد استفاده قرار گرفته است. سپس یک حلقه تکرار برای ایپاک‌های مختلف اجرا می‌شود. در هر ایپاک، داده‌های آموزشی بارگذاری شده و مدل در حالت آموزش قرار می‌گیرد. سپس خطا محاسبه می‌شود و بهینه‌سازی می‌شود. معیارهای متریکی مورد نظر نیز برای داده‌های آموزشی توسط توابعی که مستقلا پیاده‌سازی شده‌اند محاسبه می‌شوند. سپس مدل به حالت ارزیابی در آورده می‌شود و برای داده‌های تست خطا و معیارهای متریکی محاسبه می‌شود. در انتهای آموزش، این اطلاعات برای تحلیل و نمایش خروجی‌های آموزش به دست می‌آید. 

\begin{LTR}
	\begin{lstlisting}[language=Python, caption=حلقه آموزش]
num_epochs = 150
train_loss_hist = []
test_loss_hist = []
train_metrics = []
test_metrics = []

for epoch in range(num_epochs):
    loop = tqdm(train_loader)
    model.train()
    train_loss = 0.0
    train_TP, train_FP, train_TN, train_FN = 0, 0, 0, 0

    print("train")
    for inputs, labels in loop:
        inputs = inputs.to(device)
        labels = labels.to(device)
        
        outputs = model(inputs)
        loss = criterion(outputs, labels)

        optimizer.zero_grad()
        loss.backward()
        optimizer.step()
        train_loss += loss.item()

        TP, FP, TN, FN = calculate_metrics(outputs, labels)
        train_TP += TP
        train_FP += FP
        train_TN += TN
        train_FN += FN
        
        loop.set_postfix(
            epoch=epoch+1,
            total_loss=train_loss / len(train_loader),
        )
    
    train_metrics.append((train_TP, train_FP, train_TN, train_FN))
    train_loss_hist.append(train_loss / len(train_loader))
    
    model.eval()
    torch.cuda.empty_cache()
    test_loss = 0.0
    test_TP, test_FP, test_TN, test_FN = 0, 0, 0, 0
    print("Test:")
    
    with torch.no_grad():
        loop = tqdm(test_loader)
        for inputs, labels in loop:
            inputs = inputs.to(device)
            labels = labels.to(device)

            outputs = model(inputs)
            loss = criterion(outputs, labels)
            
            test_loss += loss.item()
            
            TP, FP, TN, FN = calculate_metrics(outputs, labels)
            test_TP += TP
            test_FP += FP
            test_TN += TN
            test_FN += FN
            
            loop.set_postfix(
                loss=loss.item(),
                total_loss=test_loss / len(test_loader),
            )
    test_metrics.append((test_TP, test_FP, test_TN, test_FN))
    test_loss_hist.append(test_loss / len(test_loader))
	\end{lstlisting}
\end{LTR}

\newpage
\subsubsection{نتایج MLP با تابع فعال‌ساز ReLU}

در ‎‎\autoref{fig:Q1 relu training graph}‎ نتایج مربوط به این آموزش نشان داده شده است. همانطور که مشخص است فرآیند آموزش برای مدل ‎\lr{MLP} به درستی انجام شده و مدل روی دیتاست آموزش \lr{over fit}‎ نشده است این درحالی است که میزان متریک‌ها برای هر دو دیتاست به 1 بسیار نزدیک شده است.

\begin{figure}[H]
    \centering
    \includegraphics[width=0.7\linewidth]{img/Q1_Relu_metrics_graph.png}
    \caption{نمودارهای متریک و تابع هزینه حین آموزش}
    \label{fig:Q1 relu training graph}
\end{figure}

در ‎\autoref{fig:Q1 relu result}‎ نتیجه عملکرد مدل روی دیتای ارزیابی قابل ملاحضه است و همانطور که مشخص است تنها 2 نقطه در راس بالایی مثلث اشتباه طبقه بندی شده‌اند این درحالی است که هیچ کلاسی که متعلق به داخل دایره بوده، به اشتباه به عنوان یک کلاس در خارج از دایره تشخیص داده نشده است، به عبارت دیگر
\lr{False Negative}
 برابر صفر است و  \lr{‎Precision} برابر 1 است.
 
 در انتها در ‎\autoref{fig:Q1 relu db}‎  مرز تصمیم مدل مشخص است که با توجه با این نمودار مرز حوالی راس پایین سمت راست مثلث از دقت کافی برخوردار نیست.

\begin{figure}[H] 
	\centering
	\subfigure[]{
		\includegraphics[width=0.4\linewidth]{img/Q1_relu_results.png}
		\label{fig:Q1 relu result}
		}
	\subfigure[]{
		\includegraphics[width=0.4\linewidth]{img/Q1_relu_db.png}
	    \label{fig:Q1 relu db}
		}
\caption{نتایح مدل MLP و مرز تصمیم مربوط به مدل}
\end{figure}


در ‎ \autoref{fig: Q1 relu cm}‎ماتریس درهم‌ریختگی به صورت نرمال و غیرنرمال نشان داده شده است که نشان می‌دهد مدل تنها دو نقطه را که مطعلق به داخل مثل بوده است به اشتباه به عنوان کلاس خارج از مثلث تشخیص داده است.
\begin{figure}[H] 
	\centering
	\subfigure[]{
		\includegraphics[width=0.4\linewidth]{img/Q1_relu_cm.png}
		}
	\subfigure[]{
		\includegraphics[width=0.4\linewidth]{img/Q1_relu_cmn.png}
		}
	\caption{Confusion Matrix}
	\label{fig: Q1 relu cm}
\end{figure}




\subsubsection{نتایج MLP با تابع فعال‌ساز ELU}

این مدل با استفاده از تابع ELU ساخته شده است و جزییات آن به شرج زیر می‌باشد:

\begin{LTR}
\begin{verbatim}
	MLP(
	  (layers): Sequential(
	    (0): Linear(in_features=2, out_features=8, bias=True)
	    (1): ELU(alpha=1.0)
	    (2): Linear(in_features=8, out_features=64, bias=True)
	    (3): ELU(alpha=1.0)
	    (4): Linear(in_features=64, out_features=8, bias=True)
	    (5): ELU(alpha=1.0)
	    (6): Linear(in_features=8, out_features=1, bias=True)
	    (7): Sigmoid()
	  )
	)
\end{verbatim}
\end{LTR}

در ‎‎\autoref{fig:Q1 elu training graph}‎ نتایج مربوط به این آموزش نشان داده شده است. همانطور که مشخص است فرآیند آموزش برای مدل ‎\lr{MLP} با تابع فعال‌ساز \lr{ELU}‎به درستی انجام شده و مدل روی دیتاست آموزش \lr{over fit}‎ نشده است این درحالی است که شیب نمودار هزینه همچنان کاهشی می‌باشد.در ادامه میزان متریک‌ها برای هر دو دیتاست به 1 بسیار نزدیک شده است.

\begin{figure}[H]
    \centering
    \includegraphics[width=0.7\linewidth]{img/Q1_elu_metrics_graph.png}
    \caption{نمودارهای متریک و تابع هزینه حین آموزش}
    \label{fig:Q1 elu training graph}
\end{figure}

در ‎\autoref{fig:Q1 elu result}‎ نتیجه عملکرد مدل روی دیتای ارزیابی قابل ملاحضه است و همانطور که مشخص است تنها 2 نقطه در راس بالایی مثلث اشتباه طبقه بندی شده‌اند این درحالی است که هیچ کلاسی که متعلق به داخل دایره بوده، به اشتباه به عنوان یک کلاس در خارج از دایره تشخیص داده نشده است، به عبارت دیگر
\lr{False Negative}
 برابر صفر است و  \lr{‎Precision} برابر 1 است.
 
در ‎\autoref{fig:Q1 elu db}‎  مرز تصمیم مدل مشخص است که با توجه به این نمودار و مقایسه آن با ‎\autoref{fig:Q1  relu db}‎، مشخص است که نتایج بدست آمده از این مدل توانایی تعمیم‌پذیری بیشتری دارند و هر سه راس این مثلث به یک شکل هستند.

\begin{figure}[H] 
	\centering
	\subfigure[]{
		\includegraphics[width=0.4\linewidth]{img/Q1_elu_results.png}
		\label{fig:Q1 elu result}
		}
	\subfigure[]{
		\includegraphics[width=0.4\linewidth]{img/Q1_elu_db.png}
	    \label{fig:Q1 elu db}
		}
\caption{نتایح مدل MLP و مرز تصمیم مربوط به مدل}
\end{figure}


در ‎ \autoref{fig: Q1 elu cm}‎ماتریس درهم‌ریختگی به صورت نرمال و غیرنرمال نشان داده شده است که نشان می‌دهد مدل تنها دو نقطه را که مطعلق به داخل مثل بوده است به اشتباه به عنوان کلاس خارج از مثلث تشخیص داده است.
\begin{figure}[H] 
	\centering
	\subfigure[]{
		\includegraphics[width=0.4\linewidth]{img/Q1_elu_cm.png}
		}
	\subfigure[]{
		\includegraphics[width=0.4\linewidth]{img/Q1_elu_cmn.png}
		}
	\caption{Confusion Matrix}
	\label{fig: Q1 elu cm}
\end{figure}





\subsubsection{نتایج مدل بر مبنای mcculloch-pitts نرون}

مدل‌های مبتنی بر نرون‌های mcculloch-pitts به ما این امکان را می‌دهند که با استفاده از دانش انسانی، بهترین پاسخ را برای مسئله خود بدست بیاوریم. راه حلی که می‌توانیم نقاط داخل مثلث را شناسایی کنیم حاصل ترکیب AND 3 گزاره متفاوت است؛ هر یک از این گذاره‌ها نشان‌دهنده این است که نقطه مد نظر ما نسبت به خط گذرنده از هر ضلع مثلث چه وضعیتی دارد. بدین منظور باید سه معادله خط بدست آوریم و با قرار دادن نقاط اطراف خط مشخص کنیم که تغییر علامت چه زمانی رخ می‌دهد و با چه ترکیب منطقی از این تغییر علامت‌ها می‌توانیم مثلث را پیدا کنیم.
 

در ‎‎\autoref{fig:Q1 elu training graph}‎ نتایج مربوط به این آموزش نشان داده شده است. همانطور که مشخص است فرآیند آموزش برای مدل ‎\lr{MLP} با تابع فعال‌ساز \lr{ELU}‎به درستی انجام شده و مدل روی دیتاست آموزش \lr{over fit}‎ نشده است این درحالی است که شیب نمودار هزینه همچنان کاهشی می‌باشد.در ادامه میزان متریک‌ها برای هر دو دیتاست به 1 بسیار نزدیک شده است.
روش محاسبه هر یک از این معادلات خط با قرار دادن یک جفت از راس‌های مثلث در معادله زیر می‌باشد:

\begin{align}
y &= w(x-x_1)+y_1 ,\quad w = \frac{y_1-y_2}{x_1-x_2}
\end{align}
با استفاده از رابطه فوق سه جفت مقدار برای هر معادله خط بدست آمد که این مقادیر عبارتند از
$(-2,1)$، $(2,1)$ و $(0,1)$
که سه ضلع مثلث را تشکیل می‌دهند. در ادامه یک نرون لازم است تا عملیات منطقی AND را انجام دهد. با جایگذاری نقاط مختلف از صفحه در معادلات خط بدست آمده و مقایسه آنها رابطه منطقی زیر بدست می‌آید:
$$l_1'+l_2'+l_3$$

مدل مرتبط با روابط فوق به شکل زیر در پایتون قابل پیاده‌سازی می‌باشد:
 \begin{LTR}
 	\begin{lstlisting}[language=Python, caption=حلقه آموزش]
		#define muculloch pitts
		class McCulloch_Pitts_neuron():
		
		  def __init__(self , weights ,bias, threshold):
		    self.weights = np.array(weights).reshape(-1, 1)     #define weights
		    self.threshold = threshold    #define threshold
		    self.bias = np.array(bias)
		    
		  def model(self , x):
		    #define model with threshold
		    return (x.T @ self.weights + self.bias >= self.threshold).astype(int)
		    
		def model(x):
		    neur1 = McCulloch_Pitts_neuron([-2, 1],2, 0)
		    neur2 = McCulloch_Pitts_neuron([2, 1], -6, 0)
		    neur3 = McCulloch_Pitts_neuron([0, 1], 0, 0)
		    neur4 = McCulloch_Pitts_neuron([1, 1, 1], 0, 3)
		
		    z1 = neur1.model(np.array(x))
		    z2 = neur2.model(np.array(x))
		    z3 = neur3.model(np.array(x))
		    z4 = np.squeeze(np.array([1-z1, 1-z2, z3]), axis=-1)
		    z4 = neur4.model(z4)
		
		    # 3 bit output
		    # return str(z1) + str(z2)
		    return z4
		
 	\end{lstlisting}
 \end{LTR}


همانطور که در ‎\autoref{fig:Q1 mp db}‎ مشخص است یک مثلث دقیق بدست آمده است و انتظار می‌رود که نتایج ماتریس درهم‌ریختگی آن بهترین وضعیت ممکن باشد که این مسئله در ‎\autoref{fig: Q1 mp cm}‎ قابل مشاهده است.

\begin{figure}[H] 
	\centering
		\includegraphics[width=0.4\linewidth]{img/Q1_cp_db.png}
		\label{fig:Q1 mp db}
\caption{mcculloch-pitts و مرز تصمیم مربوط به مدل}
\end{figure}


\begin{figure}[H] 
	\centering
	\subfigure[]{
		\includegraphics[width=0.4\linewidth]{img/Q1_cp_cm.png}
		}
	\subfigure[]{
		\includegraphics[width=0.4\linewidth]{img/Q1_cp_cmn.png}
		}
	\caption{Confusion Matrix}
	\label{fig: Q1 mp cm}
\end{figure}

%%%%%%%%%%%%%%%%%%%%%%%%%%%%%%%%%%%%%%%%%%%%%%%%%%%%%% Q2

\section{سوال 2}

\subsection{بخش اول}

{\large \textbf{سوال}:} 

در مورد دیتاست CWRU Bearin و عیوب موجود توضیح دهید. 

{\large \textbf{پاسخ}:} 

دیتاست استفاده شده برای این پروژه، دیتاست CWRU Bearing است. این دیتاست شامل داده‌های ارتعاشی از بلبرینگ‌ها در شرایط مختلف، شامل حالت‌های سالم و معیوب است. در این مینی‌پروژه، ما بر روی داده‌های بخش
 \lr{12k Drive End Bearing Fault Data} 
 تمرکز می‌کنیم.
علاوه بر داده‌های استفاده شده در مینی‌پروژه قبلی، ما اکنون دو کلاس جدید
 \lr{OR007@6\_1} و \lr{B007\_1}
 از بلبرینگ‌های معیوب را اضافه می‌کنیم. این گسترش پیچیدگی و تنوع بیشتری به داده‌ها اضافه می‌کند و امکان تحلیل جامع‌تر و آموزش مدل قوی‌تر را فراهم می‌کند.
انواع داده‌های معیوب و توضیحات آن به شرح زیر می‌باشد:
\begin{itemize}
    \item \textbf{سالم:} داده‌هایی که نشان‌دهنده بلبرینگ‌های در عادی خوب هستند.
    \item \textbf{کلاس معیوب 1 \lr{OR007@6\_1}:} این عیوب در قسمت بیرونی بلبرینگ رخ می‌دهد و می‌تواند ارتعاش و نویز قابل توجهی ایجاد کند که عملکرد کلی ماشین‌آلات را تحت تاثیر قرار می‌دهد.
    \item \textbf{کلاس معیوب 2 \lr{B007\_1}:} این عیوب در سطح بلبرینگ‌ها رخ می‌دهد و با نامنظمی‌هایی در سطح بلبرینگ‌ها همراه است.
    \item \textbf{کلاس معیوب 3 \lr{(IR007\_1)}:} این عیوب در قسمت داخلی بلبرینگ رخ می‌دهد و باعث ایجاد ناهنجاری در عملکرد کلی بلبرینگ می‌شود.
\end{itemize}

\subsubsection{پیش‌پردازش}

کد زیر یک چنجره به طول 
\texttt{len\_data}
را روی سری زمانی عبور می‌دهد و داده7ها را به شکل یک ماتریس 
$(n, \texttt{len\_data})$
ایجاد می‌کند که مقدار n با توجه به تعداد داده موجود در فایل دیتا بستگی دارد. سپس با توجه به متغییر ‎\texttt{n‎\_samples} تعداد نمونه دلخواه را جدا می‌کنیم. به دلیل اینکه در این مسئله تصمیم به آموزش یک شبکه عصبی داریم بیشترین تعداد داده ممکن را از دیتاست استخراج می‌کنیم که برابر 600 نمونه پنجره با طول 200 است.‎
\begin{LTR}
	\begin{lstlisting}[language=Python, caption=Sliding Window]
	n_samples = 600
	len_data = 200
	
	normal_data_matrix = normal_data[:-(normal_data.shape[0] % len_data)].reshape(-1, len_data)
	fault1_data_matrix = fault1_data[:-(fault1_data.shape[0] % len_data)].reshape(-1, len_data)
	fault2_data_matrix = fault2_data[:-(fault2_data.shape[0] % len_data)].reshape(-1, len_data)
	fault3_data_matrix = fault3_data[:-(fault3_data.shape[0] % len_data)].reshape(-1, len_data)
	\end{lstlisting}
\end{LTR}

در ادامه خروجی کد فوق نشان داده شده است.
\begin{LTR}
\begin{verbatim}
(2419, 200)
(609, 200)
(607, 200)
(612, 200)
\end{verbatim}
\end{LTR}

در ادامه با استفاده از کد زیر 
\lr{feature}های
ممکن را از هر پنجره از داده استخراج می‌کنیم و سپس ویژگی‌هایی که مد نظر هست را انتخاب می‌کنیم.

\begin{LTR}
	\begin{lstlisting}[language=Python, caption=Feature Extraction and Feature Selection]
def feature_extraction(data):
    features = {}
    # Standard Deviation
    features['Standard Deviation'] = np.std(data, axis=1)
    # Peak
    features['Peak'] = np.max(np.abs(data), axis=1)
    # Skewness
    features['Skewness'] = scipy.stats.skew(data, axis=1)
    # Kurtosis
    features['Kurtosis'] = scipy.stats.kurtosis(data, axis=1)
    # Crest Factor (peak divided by RMS)
    rms = np.sqrt(np.mean(np.square(data), axis=1))
    features['Crest Factor'] = features['Peak'] / rms
    # Clearance Factor (peak divided by the mean of the square root of the absolute values)
    features['Clearance Factor'] = features['Peak'] / np.mean(np.sqrt(np.abs(data)), axis=1)
    # Peak to Peak
    features['Peak to Peak'] = np.ptp(data, axis=1)
    # Shape Factor (RMS divided by the mean of the absolute values)
    features['Shape Factor'] = rms / np.mean(np.abs(data), axis=1)
    # Impact Factor (peak divided by mean)
    features['Impact Factor'] = features['Peak'] / np.mean(data, axis=1)
    # Square Mean Root (the square root of the mean of the squares)
    features['Square Mean Root'] = rms
    # Mean
    features['Mean'] = np.mean(data, axis=1)
    # Absolute Mean
    features['Absolute Mean'] = np.mean(np.abs(data), axis=1)
    # Root Mean Square
    features['Root Mean Square'] = rms
    # Impulse Factor (peak divided by the absolute mean)
    features['Impulse Factor'] = features['Peak'] / features['Absolute Mean']
    return features
normal_features = pd.DataFrame(feature_extraction(normal_data_matrix))
fault1_features = pd.DataFrame(feature_extraction(fault1_data_matrix))
fault2_features = pd.DataFrame(feature_extraction(fault2_data_matrix))
fault3_features = pd.DataFrame(feature_extraction(fault3_data_matrix))
normal_features['Label'] = 0
fault1_features['Label'] = 1
fault2_features['Label'] = 2
fault3_features['Label'] = 3

selected_features = ['Standard Deviation', 'Peak', 'Skewness', 'Kurtosis',
                     'Crest Factor', 'Mean', 'Root Mean Square',
                     'Impulse Factor','Square Mean Root','Shape Factor','Label']

main_df = pd.concat([normal_features, fault1_features, fault2_features, fault3_features], ignore_index=True)
df = main_df[selected_features]
	\end{lstlisting}
\end{LTR}

در ادامه تقسیم دیتاست به 3 زیرمجموعه آموزش، ارزیابی و آزمون انجام شده است که به موجب آن ابتدا به صورت تصادفی 
\lr{shuffle}
می‌شوند و پس از آن یک بار به صورت عادی و بار بعدی به صورت 
\texttt{stratify}
تقسیم داده انجام می‌شود. با مشخص کردن آرگمان 
\texttt{stratify}
باعث می‌شویم که توزیع داده در زیرمجموعه آزمون و ارزیابی بهم نریزد. تقسیم مجموعه داده با استفاده از آرگمان فوق باعث بهبود نتایج آموزش مدل نسبت به شرایطی که انتخاب هر زیرمجموعه به صورت کاملا تصادفی بوده، شده است.

اضافه کردن مجموعه داده اعتبارسنجی و بررسی تابع هزینه روی این مجموعه داده حین آموزش، موجب آن میشود که از اتفاق افتادن 
\lr{over-fitting}
جلوگیری شود.
\begin{LTR}
	\begin{lstlisting}[language=Python, caption=Train\, Validatiaon\, Test Split]
df_shuffled = df.sample(frac=1).reset_index(drop=True)

X = df_shuffled.drop('Label', axis=1)
y = df_shuffled['Label']

# X_train_raw, X_test_raw, y_train, y_test = train_test_split(X, y, test_size=0.20, random_state=53)
# X_train_raw, X_val_raw, y_train, y_val = train_test_split(X_train_raw, y_train, test_size=0.25, random_state=53)
X_train_raw, X_test_raw, y_train, y_test = train_test_split(X, y, test_size=0.20, stratify=y, random_state=53)
X_train_raw, X_val_raw, y_train, y_val = train_test_split(X_train_raw, y_train, test_size=0.25, stratify=y_train, random_state=53)
	\end{lstlisting}
\end{LTR}

\subsection{بخش دوم}
\label{section2}
 در ادامه مجموعه داده بدست آمده را نرمال می‌کنیم، باید توجه داشت که مقادیر میانگین و واریانسی که به منظور نرمال کردن مجموعه داده ارزیابی و آزمون در نظر گرفته شده است از مجموعه داده آموزش بدست آمده است. در انتها متناسب با عدد موجود در Label هر داده، یک آرایه به صورت 
One-hot
به آن داده اختصاص داده شده است.
 
\begin{LTR}
	\begin{lstlisting}[language=Python, caption=Normalization and One-hot]
 X_train, X_train_max, X_train_min = min_max_normalization(X_train_raw)
 X_test, _, _ = min_max_normalization(X_test_raw,X_train_max,X_train_min)
 X_val, _, _ = min_max_normalization(X_val_raw,X_train_max,X_train_min)
 
 X_train = torch.tensor(X_train.to_numpy(), dtype=torch.float32)
 X_val = torch.tensor(X_val.to_numpy(), dtype=torch.float32)
 X_test = torch.tensor(X_test.to_numpy(), dtype=torch.float32)
 
 y_train = torch.tensor(y_train.to_numpy(), dtype=torch.long)
 y_val = torch.tensor(y_val.to_numpy(), dtype=torch.long)
 y_test = torch.tensor(y_test.to_numpy(), dtype=torch.long)
 
 # Perform one-hot encoding
 y_train = torch.nn.functional.one_hot(y_train, num_classes=4).float()
 y_val = torch.nn.functional.one_hot(y_val, num_classes=4).float()
 y_test = torch.nn.functional.one_hot(y_test, num_classes=4).float()
	\end{lstlisting}
\end{LTR} 

در ادامه یک مدل MLP با سه لایه پنهان ایجاد می‌کنیم که شرح تعداد نرون‌های آن در کد زیر آمده است سپس با مقدار‌دهی اولیه مدل، یک شی در محیط برنامه نویسی ایجاد می‌کنیم. پارامتر بعدی که در آموزش این مدل استفاده شده است تابع هزینه 
\texttt{CrossEntropyLoss}
است که یک شی از آن با مقادیر پیش‌فرض ایجاد شده است. تابع بهینه ساز این مدل از الگوریتم Adam پیروی میکند که با گام اموزشی اولیه 
$0.001 $
شروع به بهینه کردن مدل می‌کند و در ادامه با استفاده از تابع 
\texttt{lr\_scheduler}
اگر طی 15 
Epoch
 بهبودی در نتایج تابع هزینه روی مجموعه داده ارزیابی رخ ندهد، میزان گام آموزشی 0.1 مقدار قبلی خود خواهد شد. 
\begin{LTR}
	\begin{lstlisting}[language=Python, caption=Model and Configuration]
class MLP(nn.Module):
    def __init__(self, input_dim, output_dim):
        super(MLP, self).__init__()
        self.hidden1 = nn.Linear(input_dim, 32)
        self.hidden2 = nn.Linear(32, 64)
        self.output = nn.Linear(64, output_dim)
        self.relu = nn.ReLU()
        self.softmax = nn.Softmax(dim=1)

    def forward(self, x):
        x = self.relu(self.hidden1(x))
        x = self.relu(self.hidden2(x))
        x = self.output(x)
        x = self.softmax(x)
        return x
        
# Initialize the model, loss function, and optimizer
model = MLP(input_dim=X_train.shape[1], output_dim=len(set(y)))
criterion = nn.CrossEntropyLoss()
optimizer = optim.Adam(model.parameters(), lr=0.001)
scheduler = optim.lr_scheduler.ReduceLROnPlateau(optimizer, mode='min', factor=0.1, patience=15, verbose=True)
	\end{lstlisting}
\end{LTR}         
        
در انتها مدل با استفاده از کد زیر آموزش داده می‌شود؛ باید توجه داشته که کد زیر با بررسی تغییرات تابع هزینه دستاست ارزیابی، میزان بهبود آن را بررسی می‌کند و اگر به ازای 30 
Epoch
بهبود محسوس رخ ندهد فرایند آموزش متوقف می‌شود.
\begin{LTR}
	\begin{lstlisting}[language=Python, caption=Train Loop]       
# Training loop with early stopping
num_epochs = 500
patience = 30
best_val_loss = float('inf')
epochs_without_improvement = 0
Best_model = None

# Metrics storage
train_losses = []
val_losses = []
train_accuracies = []
val_accuracies = []

for epoch in range(num_epochs):
    model.train()
    train_loss = 0.0
    train_correct = 0
    total_train = 0

    for inputs, labels in train_loader:
        optimizer.zero_grad()
        outputs = model(inputs)
        loss = criterion(outputs, labels)
        loss.backward()
        optimizer.step()
        train_loss += loss.item()

        _, preds = torch.max(outputs, 1)
        _, targets = torch.max(labels,1)
        train_correct += (preds == targets).sum().item()
        total_train += labels.size(0)

    train_loss /= len(train_loader)
    train_accuracy = train_correct / total_train
    train_losses.append(train_loss)
    train_accuracies.append(train_accuracy)

    model.eval()
    val_loss = 0.0
    val_correct = 0
    total_val = 0
    with torch.no_grad():
        for inputs, labels in val_loader:
            outputs = model(inputs)
            loss = criterion(outputs, labels)
            val_loss += loss.item()

            _, preds = torch.max(outputs, 1)
            _, targets = torch.max(labels,1)
            val_correct += (preds == targets).sum().item()
            total_val += labels.size(0)

    val_loss /= len(val_loader)
    val_accuracy = val_correct / total_val
    val_losses.append(val_loss)
    val_accuracies.append(val_accuracy)

    print(f'Epoch {epoch+1}/{num_epochs}, Train Loss: {train_loss:.4f}, Val Loss: {val_loss:.4f}, Train Accuracy: {train_accuracy:.4f}, Val Accuracy: {val_accuracy:.4f}')

    scheduler.step(val_loss)
    
    if  best_val_loss - val_loss > 1e-3:
        best_val_loss = val_loss
        torch.save(model.state_dict(), 'best_model.pth')
        epochs_without_improvement = 0
    else:
        epochs_without_improvement += 1

    if epochs_without_improvement >= patience:
        print(f'Early stopping triggered after {epoch+1} epochs.')
        break
	\end{lstlisting}
\end{LTR}         
        
\subsubsection{نتایج}
پس از آموزش مدل با 
configuration 
که در بالا مطرح شد نتایج بدست آمده از تابع هزینه و متریک 
Accuracy
در 
\autoref{fig:Q2 result basic}
نشان داده شده است. در این آموزش، با گذشتن 258 
Epoch
بهبودی در میزان تابع هزینه دیتاست ارزیابی مشاهده نشده است و فرایند آموزش
early stop 
شده است. همانطور که مشخص است میزان accuracy این مجموعه داده آموزش و ارزیابی در طی فرایند آموزش به حدود یک نزدیک شده است و همانطور که از این نمودار مشخص است، بهینه سازی این مدل در 2 نقطه به 
local extremum 
رسیده است و برای حدود چند 
Epoch
بهبودی در نتایج این متریک مشاهده نمی‌شود. 
\begin{figure}[H]
\centering
\includegraphics[width=1\linewidth]{"img/Q2/result basic"}
\caption{نتایج آموزش مدل MLP}
\label{fig:Q2 result basic}
\end{figure}

به منظور تحلیل بهتر نتایح بدست آمده از آموزش این مدل متریک‌ها و معیارهای دیگری علاوه بر 
Accuracy 
مورد نیاز است زیرا که متریک 
Accuracy 
با در نظر گرفتن کلاس‌های نرمالی که به درستی تشخیص داده شده‌اند در فرمول خود، باعث کاهش اثر نمونه‌هایی که به اشتباه تشخیص داده‌شده‌اند می‌شود بنابراین متریک‌های دیگری مورد نیاز است که در ادامه نتایج مدل را برحسب 
Recall و Precision و F-1 Score
و ماتریس درهم‌ریختگی روی مجموعه داده آزمون بررسی می‌شود.

نتایج آموزش مدل روی داده آزمون در 
\autoref{fig:Q2 cm basic}
و
\autoref{Q2 basic result}
نشان دهنده عملکرد بسیار خوب این مدل می‌باشد که تمام نمونه‌ها را به درستی تشخیص داده است.
 
\begin{LTR}
\label{Q2 basic result}
\begin{verbatim}
              precision    recall  f1-score   support

           0       1.00      1.00      1.00       359
           1       1.00      1.00      1.00       375
           2       1.00      1.00      1.00       348
           3       1.00      1.00      1.00       358

    accuracy                           1.00      1440
   macro avg       1.00      1.00      1.00      1440
weighted avg       1.00      1.00      1.00      1440
\end{verbatim}
\end{LTR}

\begin{figure}[H]
\centering
\includegraphics[width=0.5\linewidth]{"img/Q2/cm basic"}
\caption{Confiusion Matrix}
\label{fig:Q2 cm basic}
\end{figure}

\subsection{بخش سوم}

در ادامه با استفاده از کد زیر 2 مدل جدید با 
configuratinهای 
جدید ایجاد می‌کنیم که در مدل اول تنها تابع هزینه به 
\texttt{L1Loss}
تبدیل شده و در مدل دوم علاوه بر آن بهینه‌ساز به الگوریتم 
\texttt{SGD}
تغییر یافته است.
\begin{LTR}  
	\begin{lstlisting}       
# Model 1
model = MLP(input_dim=X_train.shape[1], output_dim=len(set(y)))
criterion = nn.L1Loss()
optimizer = optim.Adam(model.parameters(), lr=0.1)
scheduler = optim.lr_scheduler.ReduceLROnPlateau(optimizer, mode='min', factor=0.1, patience=15, verbose=True)

# model 2
model = MLP(input_dim=X_train.shape[1], output_dim=len(set(y)))
criterion = nn.L1Loss()
optimizer = optim.SGD(model.parameters(), lr=0.1)
scheduler = optim.lr_scheduler.ReduceLROnPlateau(optimizer, mode='min', factor=0.1, patience=15, verbose=True)
	\end{lstlisting}
\end{LTR}         
        
\subsubsection{نتایج}

 با توجه به نمودارهای موجود در 
\autoref{fig:Q2 metric-of-opt-loss} و \autoref{fig:Q2 loss-of-opt-loss}
مشخص می‌شود که تغییر تابع هزینه به تنهایی می‌تواند به شدت روی عملکرد مدل تاثیر بگذارد؛ این مسئله با بررسی نمودارهای نارنجی که مربوط به مدل با تابع هزینه
\texttt{L1Loss} و 
بهینه ساز
\texttt{Adam}
است مشخص می‌شود. در آموزش این مدل مشخص می‌شود که یک نقطه بهینه محلی پیدا شده است که مدل نتوانسته است از آن خارج شود و مقدار 
Accuracy
برابر 75‎\% شده است که نشان دهنده افت شدید این متریک نسبت به شرایطی است که تابع هزینه از نوع
\texttt{CrossEntropyLoss}
است.

در ادامه با تغییر بهینه‌ساز در فرایند آموزش به 
\texttt{SGD}
نشان داده می‌شود که دوباره مدل به نتیجه مطلوب می‌رسد اما باید این نکته را در نظر داشته که بابت رسیدن به این نتیجه نیاز است که 500 
Epoch
طی شود که خیلی بیشتر از تعداد 
Epoch
طی شده در فرایند آموزش نشان داده‌شده در 
\autoref{section2}
 می‌باشد. نکته حائز اهمیت در نمودارهای ‎
 \autoref{fig:Q2 loss-of-opt-loss}
 این است که آموزشی که با نمودار آبی نشان داده شده است همچنان می‌تواند آموزش داده شود و داری شیب نزولی هم در نمودار آموزش و هم در نمودار ارزیابی است.
\begin{figure}[H]
\centering
\includegraphics[width=1\linewidth]{"img/Q2/loss of opt loss"}
\caption{تابع هزینه آموزش مدل MLP با بهینه ساز و تابع هزینه جدید}
\label{fig:Q2 loss-of-opt-loss}
\end{figure}

\begin{figure}[H]
\centering
\includegraphics[width=1\linewidth]{"img/Q2/metric of opt loss"}
\caption{متریک‌های آموزش مدل MLP با بهینه ساز و تابع هزینه جدید}
\label{fig:Q2 metric-of-opt-loss}
\end{figure}

در ادامه به منظور مقایسه عملکرد مدلی که تابع هزینه و الگوریتم بهینه‌سازی آن عوض شده است با مدل قبلی، ماتریس درهم‌ریختگی را روی مجموعه داده تست محاسبه می‌کنیم که در ‎
\autoref{fig:Q2 cm-of-opt-loss}
نمایش آن قابل ملاحظه است. همانطور که از این ماتریس مشخص است، مدلی که تابع هزینه و بهینه ساز آن تغییر کرده است در 4 نمونه از کلاس سالم را به اشتباه متعلق به کلاس 2 تشخیص داده است و 1 نمونه از کلاس 3 را متعلق به کلاس 1 تشخیص داده است که در مجموع 5 خطا دارد؛ این درحالی است که مدل بدست آمده در
\autoref{section2}
هیچ خطایی روی محموعه داده آزمون نداشته است، بنابراین تغییر تابع هزینه و یا بهینه ساز می‌تواند موجب کاهش عملکرد مدل در این مسئله بشود.  
\begin{figure}[H]
\centering
\includegraphics[width=0.5\linewidth]{"img/Q2/cm of opt loss"}
\caption{ماتریس درهم‌ریختگی روی داده‌های آزمون}
\label{fig:Q2 cm-of-opt-loss}
\end{figure}

\subsection{بخش چهارم}

K-fold Cross-validation
یک روش آماری قوی برای ارزیابی عملکرد و تعمیم‌پذیری یک مدل یادگیری ماشین است. در این تکنیک، مجموعه داده به \(k\) بخش مساوی تقسیم می‌شود. مدل بر روی \(k-1\) بخش آموزش داده می‌شود و بر روی بخش باقی‌مانده ارزیابی می‌شود. این فرایند \(k\) بار تکرار می‌شود، هر بار با استفاده از یک بخش متفاوت برای تست. سپس نتایج میانگین گرفته می‌شود تا یک اندازه‌گیری جامع از عملکرد مدل ارائه شود. این روش تضمین می‌کند که هر  داده هم برای آموزش و هم برای ارزیابی استفاده می‌شود و باعث کاهش bias و واریانس در مقایسه با تقسیمات ساده آموزش و ارزیابی می‌شود. K-fold Cross-validation به‌ویژه در مجموعه داده‌های کوچک که حفظ حداکثر داده‌ها برای آموزش اهمیت دارد، مفید است.

Stratified K-Fold Cross-Validation
 یک نوع تغییر یافته از K-fold Cross-validation است که هدف آن حفظ توزیع متغیر هدف در تمام بخش‌هاست. در مسائل دسته‌بندی، این روش تضمین می‌کند که هر بخش تقریباً همان نسبت کلاس‌ها را دارد که در کل مجموعه داده وجود دارد. این طبقه‌بندی برای مجموعه داده‌های imbalance که برخی کلاس‌ها ممکن است تعداد کمی داشته باشند، بسیار مهم است. با حفظ توزیع کلاس‌ها، Stratified K-Fold Cross-Validation یک تخمین دقیق‌تر از عملکرد مدل، به‌ویژه برای کلاس‌های اقلیت ارائه می‌دهد. این روش به جلوگیری از نتایج گمراه‌کننده کمک می‌کند و آن را به انتخابی بهتر برای آموزش مدل‌های یادگیری ماشین تبدیل می‌کند.



با استفاده از کد زیر می‌‌توانیم مجموعه داده را به 
k - fold
تقسیم کنیم. نکته حائز اهمیت این است که مجموعه داده 
test
به دلیل اینکه قابلیت مقایسه با آموزش‌های قبلی حفظ شود در تمام آموزش‌ها در طی حل سوال 2 یکسان بوده است.

\begin{LTR}  
	\begin{lstlisting}       
kf = KFold(n_splits=5)
kf_results = []

for fold, (train_index, val_index) in enumerate(kf.split(X_tensor)):
    print(f"Fold {fold+1}")
    X_train, X_val = X_tensor[train_index], X_tensor[val_index]
    y_train, y_val = y_tensor[train_index], y_tensor[val_index]
    
    train_data = torch.utils.data.TensorDataset(torch.tensor(X_train, dtype=torch.float32), torch.tensor(y_train, dtype=torch.float))
    val_data = torch.utils.data.TensorDataset(torch.tensor(X_val, dtype=torch.float32), torch.tensor(y_val, dtype=torch.float))
    
    train_loader = torch.utils.data.DataLoader(train_data, batch_size=512, shuffle=True)
    val_loader = torch.utils.data.DataLoader(val_data, batch_size=2000, shuffle=False)
    
    model = MLP(input_dim=X_train.shape[1], output_dim=len(set(y))).to(device)
    optimizer = optim.Adam(model.parameters(), lr=0.001)
    scheduler = optim.lr_scheduler.ReduceLROnPlateau(optimizer, mode='min', factor=0.1, patience=15, verbose=True)
	\end{lstlisting}
\end{LTR}    

\subsubsection{نتایج}
در 
\autoref{fig:five_images}
نمودارهای ماتریس درهم‌ریختگی مشخص شده است، با استفاده از این نمودارها می‌توان متریک‌های 
Recall و Precision
را محاسبه کرد. با توجه به اینکه این نمودارها روی مجموعه داده آزمون حساب شده است، و این مجموعه داده در طول آموزش‌ها یکسان بوده است، می‌توان نتیجه گرفت که در قسمت‌های قبل نتایج صورت تصادفی کسب نشده‌اند و مدل به درستی روی دیتاست آموزش دیده است. اگر پیچیدگی دیتاست بیشتر بود با انجام اینکار ممکن بود نتایج در یک 
fold
به خصوص بهتر از بقیه آموزش‌ها شود.
\begin{figure}[H]
    \centering
    \begin{subfigure}[b]{0.33\textwidth}
        \centering
        \includegraphics[width=\textwidth]{img/Q2/f1.png}
        \label{fig:image1}
    \end{subfigure}
    \hfill
    \begin{subfigure}[b]{0.32\textwidth}
        \centering
        \includegraphics[width=\textwidth]{img/Q2/f2.png}
        \label{fig:image2}
    \end{subfigure}
    \hfill
    \begin{subfigure}[b]{0.33\textwidth}
        \centering
        \includegraphics[width=\textwidth]{img/Q2/f3.png}
        \label{fig:image3}
    \end{subfigure}
    
    \vspace{0.5cm}
    
    \begin{subfigure}[b]{0.33\textwidth}
        \centering
        \includegraphics[width=\textwidth]{img/Q2/f4.png}
        \label{fig:image4}
    \end{subfigure}
    \hfill
    \begin{subfigure}[b]{0.33\textwidth}
        \centering
        \includegraphics[width=\textwidth]{img/Q2/f5.png}
        \label{fig:image5}
    \end{subfigure}
    
    \caption{5 fold cross-validation confusion matrix}
    \label{fig:five_images}
\end{figure}

در 
\autoref{fig:five_images_m}
 تصاویر مربوط تابع هزینه و متریک‌ 
Accuracy 
طی فرایند آموزش نمایش داده نشان داده شده است که طی fold های متفاوت تفاوت فاحشی با یکدیگر ندارند و با توجه به اینکه حلقه آموزش شرایط
early stop
را بررسی می‌کرده، همه آموزش‌ها در حدود 
epoch
260 
به اتمام رسیده است.


\begin{figure}[H]
    \centering
    \begin{subfigure}[b]{0.45\textwidth}
        \centering
        \includegraphics[width=\textwidth]{img/Q2/trainlf.png}
        \label{fig:image1}
    \end{subfigure}
    \hfill
    \begin{subfigure}[b]{0.45\textwidth}
        \centering
        \includegraphics[width=\textwidth]{img/Q2/vallf.png}
        \label{fig:image3}
    \end{subfigure}
    
    \vspace{0.5cm}
    
    \begin{subfigure}[b]{0.45\textwidth}
        \centering
        \includegraphics[width=\textwidth]{img/Q2/trainmf.png}
        \label{fig:image4}
    \end{subfigure}
    \hfill
    \begin{subfigure}[b]{0.45\textwidth}
        \centering
        \includegraphics[width=\textwidth]{img/Q2/valmf.png}
        \label{fig:image5}
    \end{subfigure}
    
    \caption{5 fold cross-validation metrics}
    \label{fig:five_images_m}
\end{figure}

\section{سوال 3}

در ادامه حل این سوال مجموعه داده دارو درنظر گرفته شده است و الگوریتم‌ها به روی این مجموعه داده اجرا شده‌اند.
در ادامه تحلیل آماری مجموعه داده دارو انجام شده است. همانطور که در 
\autoref{fig:Q3 stat}
مشخص شده است این دیتاست تا حدی 
imbalance
است و فراوانی داروی x و y بیشتر از باقی داروها می‌باشد اعداد دقیق مربوط به فراوانی این دیتاست در ‎
\autoref{Q3 stat}
نشان‌داده‌شده است.
\begin{figure}[H]
\centering
\includegraphics[width=1\linewidth]{img/Q3/stat}
\caption{نمودار توزیع داده}
\label{fig:Q3 stat}
\end{figure}

\begin{LTR}
\label{Q3 stat}
\begin{verbatim}
The number of unique values in the column "Drug" is 5
drugY    91
drugX    54
drugA    23
drugC    16
drugB    16
Name: Drug, dtype: int64

\end{verbatim}
\end{LTR}

\autoref{tab:Q3 Table}
نشان‌دهنده چند سطر از مجموعه داده دارو می‌باشد که مقادیر توصیفی ستون‌های 
Sex، BP و Cholesterol 
به شرح زیر می‌باشد:
\begin{LTR}
\label{Q3 stat}
\begin{verbatim}
Values of Sex column =  ['F' 'M']
Values of BP column =  ['HIGH' 'LOW' 'NORMAL']
Values of Cholesterol column ['HIGH' 'NORMAL']
\end{verbatim}
\end{LTR}


\begin{table}[H]
\centering
\begin{tabular}{ccccccc}
\toprule
 & Age & Sex & BP & Cholesterol & Na\_to\_K & Drug \\
\midrule
0 & 23 & F & HIGH & HIGH & 25.355 & drugY \\
1 & 47 & M & LOW & HIGH & 13.093 & drugC \\
2 & 47 & M & LOW & HIGH & 10.114 & drugC \\
3 & 28 & F & NORMAL & HIGH & 7.798 & drugX \\
4 & 61 & F & LOW & HIGH & 18.043 & drugY \\
\bottomrule
\end{tabular}
\caption{Sample Data}
\label{tab:Q3 Table}
\end{table}

\subsection{بخش اول}



{\large \textbf{سوال}:} 

توضیح دهید که از چه روشی برای انتخاب بخشی از داده استفاده کرده‌اید. آیا روش بهتری برای این کار می‌شناسید؟



{\large \textbf{سوال}:} 

به منظور پاسخ داده به این سوال ابتدا باید کدی که به وسیله آن دیتاست تقسیم شده است مورد بررسی قرار بگیرد.
کد زیر به منظور تقسیم دیتاست موجود به دو زیرمجموعه آموزش و ارزیابی است که در نتیجه آن
40\% 
 از این دیتاست با توزیع یکنواخت به مجموعه ارزیابی تخصیص داده می‌شود.

\begin{LTR}
	\begin{lstlisting}[language=Python, caption=Data Selection]
		X_train, X_test, y_train, y_test = train_test_split(X, y, test_size=0.4, random_state=53)
	\end{lstlisting}
\end{LTR}

نکته حائز اهمیت این است که این دیتاست از تفکیک پذیری بالایی برخوردار است و نیازی نیست که از روش‌های پیشرفته‌تر به منظور تقسیم آن استفاده کنیم اما با توجه به 
imbalance
بودن این دیتاست می‌توان از روش 
stratify
به منظور حفظ توزیع داده‌ها و یا روش
K-Fold-cross-validation
استفاده کرد تا بهترین نتیجه ممکن را بدست آورد و از عملکرد مدل مطمئن شد.

در ادامه با استفاده از کد زیر و 
Configuration
پیش‌فرض کتابخانه
sklearn
یک درخت تصمیم ایجاد شده است و روی دیتاست آموزش، آموزش دیده است. نتایج بدست آمده از این آموزش در 
\autoref{fig Q3 :basic-tree}
نشان داده شده است.
\begin{LTR}
	\begin{lstlisting}[language=Python, caption=Train Decision Tree]
# Create a Decision Tree Classifier
clf = DecisionTreeClassifier(random_state=53)

# Train the classifier on the training data
clf.fit(X_train, y_train)

plot_tree(clf, filled=True, feature_names=X.columns, class_names=clf.classes_, rounded=True)
plt.title('Decision Tree')
plt.show()
	\end{lstlisting}
\end{LTR}


\begin{figure}[H]
\centering
\includegraphics[width=1\linewidth]{"img/Q3/basic tree"}
\caption{}
\label{fig Q3 :basic-tree}
\end{figure}

\subsection{بخش دوم}
با توجه به 
‎\autoref{fig Q3 :basic-tree}
 می‌توان مشخص کرد که از داده‌های جنسیت به منظور تعیین کلاس هر نمونه استفاده نشده است. به منظور تحلیل این مسئله لازم است تا پراکندگی جنسیت، نسبت به هر نوع دارو رسم شود. این نمودار در 
\autoref{fig:drug-vs-sex}
 رسم شده و است و همانطور که مشخص است میزان مردان نسبت به زنان همواره از 
 entropy
 بالایی برخوردار بوده که به همین دلیل نیز جنسیت در نوع دارو تاثیر چندانی نگذاشته است، به بیان دیگر این بیماری با جنسیت وابستگی پایینی دارد.
 
 
\begin{figure}[H]
\centering
\includegraphics[width=0.5\linewidth]{"img/Q3/drug vs sex"}
\caption{Sex over Drug}
\label{fig:drug-vs-sex}
\end{figure}

در ادامه متریک‌‌های موردنیاز به منظور بررسی نتایج مدل محاسبه شده است. همانطور که مشخص است برای دو زیرمجموعه آموزش و ارزیابی تمامی متریک‌های و ماتریس درهم‌ریختگی در 
\autoref{fig: Q3 basic cm}
مقداری برابر 1 دارند که این به معنی عملکرد بی‌عیب مدل ما است.
\begin{LTR}
\begin{verbatim}
training metrics:
Classification Report:
               precision    recall  f1-score   support

       drugA       1.00      1.00      1.00        17
       drugB       1.00      1.00      1.00        11
       drugC       1.00      1.00      1.00        11
       drugX       1.00      1.00      1.00        38
       drugY       1.00      1.00      1.00        63

    accuracy                           1.00       140
   macro avg       1.00      1.00      1.00       140
weighted avg       1.00      1.00      1.00       140

test metrics:

Classification Report:
               precision    recall  f1-score   support

       drugA       1.00      1.00      1.00         6
       drugB       1.00      1.00      1.00         5
       drugC       1.00      1.00      1.00         5
       drugX       1.00      1.00      1.00        16
       drugY       1.00      1.00      1.00        28

    accuracy                           1.00        60
   macro avg       1.00      1.00      1.00        60
weighted avg       1.00      1.00      1.00        60
\end{verbatim}
\end{LTR}

\begin{figure}[H] 
	\centering
	\subfigure[]{
		\includegraphics[width=0.4\linewidth]{img/Q3/Basic train.png}
		}
	\subfigure[]{
		\includegraphics[width=0.4\linewidth]{img/Q3/Basic test.png}
		}
	\caption{Confusion Matrix}
	\label{fig: Q3 basic cm}
\end{figure}

حال با استفاده از 
\texttt{GridSearchCV}
هایپرپارامتر‌های متفاوت را با استفاده از متد
5-fold-corss-validation
بررسی می‌کنیم تا بهترین ساختار مدل را پیدا کنیم. در این جستجو مقادیر
\texttt{max\_depth}، \texttt{min\_samples\_split} و \texttt{min\_samples\_leaf}
را در یک بازه متناسب با دیتاستی که در اختیار داریم تغییر می‌دهیم. کد زیر به منظور آموزش مدل‌های مختلف استفاده شده است و در انتها بهترین هایپرپارامترها را خروجی می‌دهد.
\begin{LTR}
	\begin{lstlisting}[language=Python, caption=Grid search]
param_grid = {
    'max_depth': [2, 3, 4, 5, 10, 15 ],
    'min_samples_split':  [2, 5, 10, 15],
    'min_samples_leaf': [1, 2, 4, 7, 10]
}
scoring = {
    'precision': make_scorer(precision_score, average='macro'),
    'recall': make_scorer(recall_score, average='macro')
}
# Perform grid search with cross-validation
grid_search = GridSearchCV(estimator=clf, param_grid=param_grid,
                           scoring=scoring, refit='precision', return_train_score=True)
grid_search.fit(X_train, y_train)
	\end{lstlisting}
\end{LTR}

نتایج بدست آمده از این گروه آموزش‌ها نشان می‌دهد که مدل بهینه هایپرپارامتر‌های زیر را دارد.

\begin{LTR}
\begin{verbatim}
Best Parameters: {'max_depth': 4, 'min_samples_leaf': 1, 'min_samples_split': 2}
\end{verbatim}
\end{LTR}

نمودارهای موجود در 
\autoref{fig:Q3 hyper}
به ازای تغییرات هر هایپرپارامتر کشیده شده است. همانطور که مشخص است، متغییر 
\texttt{min\_samples\_split}
با افزایش مقدار، گاهی موجب بهبود و گاهی موجب افت عملکرد مدل می‌شود که این بهبود یا افت به مقدار دو متغییر دیگر بستگی دارد بنابراین نمیتوان به صورت کلی نظر قطعی درمورد این متغییر داد.
متغییر 
\texttt{max\_depth}
یک رفتار تکرار شونده را نشان می‌دهد که به غیر از مقادیر خاص، نمودار آن افزایشی می‌باشد بنابراین افزایش مقدار این متغییر می‌تواند به بهبود مدل کمک کند.
در انتها متغییر
\texttt{min\_samples\_leaf}
و نمودار مربوط به آن نشان می‌دهد که با افزایش مقدار این متغییر عملکرد مدل در این مسئله به خصوص کاهش پیدا می‌کند بنابراین کمترین مقدار ممکن را برای این متغییر درنظر می‌گیریم. 

\begin{figure}[H]
\centering
\includegraphics[height=0.9\textheight]{img/Q3/hyper}
\caption{Hyper-parameters Graph}
\label{fig:Q3 hyper}
\end{figure}

حال به سراغ بررسی نتایج بدست آمده از بهترین مدل می‌رویم، همانطور که در 
\autoref{fig:best-tree}
مشخص است، تفاوتی با درخت موجود در 
\autoref{fig Q3 :basic-tree}
نداشته است، از طرف دیگر با توجه به متریک‌ها و ماتریس درهم‌ریختگی در 
\autoref{fig: Q3 best cm}
همچنان عملکرد مدل در بهترین حالت خود می‌باشد و تمام نمونه‌ها را به درستی تشخیص داده است
\begin{figure}[H]
\centering
\includegraphics[width=0.7\linewidth]{"img/Q3/Best tree"}
\caption{Best Tree}
\label{fig:best-tree}
\end{figure}

\begin{LTR}
\begin{verbatim}
training metrics:
Classification Report:
               precision    recall  f1-score   support

       drugA       1.00      1.00      1.00        17
       drugB       1.00      1.00      1.00        11
       drugC       1.00      1.00      1.00        11
       drugX       1.00      1.00      1.00        38
       drugY       1.00      1.00      1.00        63

    accuracy                           1.00       140
   macro avg       1.00      1.00      1.00       140
weighted avg       1.00      1.00      1.00       140

test metrics:

Classification Report:
               precision    recall  f1-score   support

       drugA       1.00      1.00      1.00         6
       drugB       1.00      1.00      1.00         5
       drugC       1.00      1.00      1.00         5
       drugX       1.00      1.00      1.00        16
       drugY       1.00      1.00      1.00        28

    accuracy                           1.00        60
   macro avg       1.00      1.00      1.00        60
weighted avg       1.00      1.00      1.00        60
\end{verbatim}
\end{LTR}

\begin{figure}[H] 
	\centering
	\subfigure[]{
		\includegraphics[width=0.4\linewidth]{img/Q3/Basic train.png}
		}
	\subfigure[]{
		\includegraphics[width=0.4\linewidth]{img/Q3/Basic test.png}
		}
	\caption{Confusion Matrix}
	\label{fig: Q3 best cm}
\end{figure}

در انتها به سراغ 
post-pruning
می‌رویم و با استفاده از کد زیر چند 
$\alpha$
متفاوت را تست می‌کنیم که نتیجه آن در 
\autoref{fig:alpha}
قابل مشاهده است. در نتیجه 
post-pruning 
روی درخت به وجود آمده توست مجموعه داده دارو تاثیر منفی خواهد داشت.
\begin{LTR}
	\begin{lstlisting}[language=Python, caption=best alpha]
clf = DecisionTreeClassifier(random_state=53)
path = clf.cost_complexity_pruning_path(X_train, y_train)
ccp_alphas, impurities = path.ccp_alphas, path.impurities
clfs = []
for ccp_alpha in ccp_alphas:
    clf = DecisionTreeClassifier(random_state=53, ccp_alpha=ccp_alpha)
    clf.fit(X_train, y_train)
    clfs.append(clf)
	\end{lstlisting}
\end{LTR}
\begin{figure}[H]
\centering
\includegraphics[width=0.7\linewidth]{img/Q3/alpha}
\caption{}
\label{fig:alpha}
\end{figure}


\subsection{بخش سوم}
AdaBoost و RandomForest
 دو تکنیک یادگیری جمعی هستند که می‌توانند عملکرد درخت تصمیم را به طور قابل توجهی بهبود بخشند. این تکنیک‌ها به شرح زیر عمل می‌کنند.
 
{\large \textbf{AdaBoost}:} 
AdaBoost 
یک تکنیک تقویت است که بر بهبود دقت پیش‌بینی‌ها با ترکیب چندین درخت تصمیم تکیه دارد. ایده کلیدی پشت AdaBoost
 تنظیم وزن داده‌های آموزشی بر اساس عملکرد درخت قبلی است. 
 
طی فرایند آموزش یک مدل 
AdaBoost
 ابتدا، وزن‌ تمام نمونه‌ها برابر یکدیگر قرار داده می‌شود. سپس یک درخت تصمیم‌گیری با استفاده از این داده‌هاآموزش داده می‌شود. در ادامه، وزن نمونه‌هایی که نادرست طبقه‌بندی‌شده، افزایش می‌یابد تا درخت بعدی بر روی آن‌ها تمرکز کند. این فرایند برای تعداد مشخصی از تکرارها یا تا رسیدن به دقت مطلوب تکرار می‌شود. مدل نهایی یک جمع وزن‌دار از مجموعه درخت‌‌ها است، که در آن درخت‌های دقیق‌تر وزن بیشتری دارند. فرایند آموزش یک مدل AdaBoost در 
 \autoref{fig:ada}
 آورده شده است.
 
 
\begin{figure}[H]
\centering
\includegraphics[width=1\linewidth]{img/Q3/ada}
\caption{نحوه عملکرد و آموزش مدل AdaBoost \cite{towardsdatascience2020ensemblelearning}}
\label{fig:ada}
\end{figure}

{\large \textbf{RandomForest}:} 
RandomForest 
یک  است که با یک مجموعه از درختان تصمیم، یک جنگل تصمیم می‌سازد که هر کدام از درختان این جنگل بر روی یک زیرمجموعه تصادفی از داده‌ها و ویژگی‌ها آموزش داده می‌شوند. 

در این روش، ابتدا زیرمجموعه‌های تصادفی از داده‌های آموزشی انتخاب می‌شوند. سپس یک درخت تصمیم‌گیری بر روی هر زیرمجموعه آموزش داده می‌شود. هر درخت به طور مستقل یک پیش‌بینی انجام می‌دهد و پیش‌بینی نهایی با رأی اکثریت بین تمامی درختان انجام می‌شود. این روش مؤثر است زیرا درختان را با آموزش آن‌ها بر روی نمونه‌ها و زیرمجموعه‌های ویژگی‌های مختلف، از هم جدا می‌کند. تجمیع پیش‌بینی‌ها از چندین درخت به طور کلی به عملکرد بهتری نسبت به هر درخت منفرد منجر می‌شود.

هر دو روش AdaBoost و RandomForest عملکرد درخت تصمیم‌گیری را بهبود می‌بخشند. در حالی که AdaBoost بر بهبود متوالی و تنظیم وزن‌ها تمرکز دارد، RandomForest مجموعه‌ای متنوع از درختان را می‌سازد و نتایج آن‌ها را میانگین می‌گیرد. این دو تکنیک می‌توانند مدلی بهتر ایجاد کنند.
\cite{geeksforgeeks2023differences}

حال با استفاده از کد زیر یک مدل
RandomForest
را آموزش می‌دهیم.


\begin{LTR}
\begin{lstlisting}[language=Python, caption=Random Forest Training]
# Define the parameter grid for the Random Forest
param_grid_rf = {
    'n_estimators': [100, 200, 300],
    'max_features': ['auto', 'sqrt'],
    'max_depth': [4, 6, 8, None],
    'min_samples_split': [2, 5, 10],
    'min_samples_leaf': [1, 2, 4],
    'bootstrap': [True, False]
}

# Initialize the Random Forest classifier
rf_clf = RandomForestClassifier(random_state=53)

# Initialize GridSearchCV
grid_search_rf = GridSearchCV(estimator=rf_clf, param_grid=param_grid_rf, cv=5, n_jobs=-1, )

# Fit GridSearchCV
grid_search_rf.fit(X_train, y_train)

\end{lstlisting}
\end{LTR}

نتایج بدست آمده از مدل فوق به شرح زیر می‌باشد که تمام متریک‌ها برابر 1 شده‌اند و از روی ماتریس درهم‌ریختگی 
\autoref{fig: Q3 rf cm}
 مشخص است که تمام نمونه‌ها به درستی تشخیص داده‌شده‌اند.


\begin{LTR}
\begin{verbatim}
training metrics:
Classification Report:
               precision    recall  f1-score   support

       drugA       1.00      1.00      1.00        17
       drugB       1.00      1.00      1.00        11
       drugC       1.00      1.00      1.00        11
       drugX       1.00      1.00      1.00        38
       drugY       1.00      1.00      1.00        63

    accuracy                           1.00       140
   macro avg       1.00      1.00      1.00       140
weighted avg       1.00      1.00      1.00       140

test metrics:

Classification Report:
               precision    recall  f1-score   support

       drugA       1.00      1.00      1.00         6
       drugB       1.00      1.00      1.00         5
       drugC       1.00      1.00      1.00         5
       drugX       1.00      1.00      1.00        16
       drugY       1.00      1.00      1.00        28

    accuracy                           1.00        60
   macro avg       1.00      1.00      1.00        60
weighted avg       1.00      1.00      1.00        60
\end{verbatim}
\end{LTR}

\begin{figure}[H] 
	\centering
	\subfigure[]{
		\includegraphics[width=0.4\linewidth]{img/Q3/Basic train.png}
		}
	\subfigure[]{
		\includegraphics[width=0.4\linewidth]{img/Q3/Basic test.png}
		}
	\caption{Confusion Matrix}
	\label{fig: Q3 rf cm}
\end{figure}


\section{سوال 4}

روش Bayse با تکیه بر توضیح احتمالاتی مسئله و با فرض اینکه رگرسورها از یکدیگر مستقل هستند، از قانون
Bayse 
استفاده می‌کند تا یک مدل طبقه‌بندی را ایجاد کند. با توجه به اینکه الگوریتم به توزیع احتمالاتی وابسته است نرمال کردن دیتاست تاثیری در نتیجه نخواهد داشت بنابراین پیش‌پردازش خاصی لازم نیست. همانطور که در ‎
\autoref{tab:Q4 data}
مشخص است این دیتاست 13 ویژگی را اندازه‌گیری کرده است تا به وسیله آنها پیشبینی انجام شود.

\begin{table}[ht]
\centering
\begin{tabular}{ccccccccccccccc}
\toprule
 & Age & Sex & CP & Trestbps & Chol & FBS & Restecg & Thalach & Exang & Oldpeak & Slope & CA & Thal & Target \\
\midrule
0 & 52 & 1 & 0 & 125 & 212 & 0 & 1 & 168 & 0 & 1.0 & 2 & 2 & 3 & 0 \\
1 & 53 & 1 & 0 & 140 & 203 & 1 & 0 & 155 & 1 & 3.1 & 0 & 0 & 3 & 0 \\
2 & 70 & 1 & 0 & 145 & 174 & 0 & 1 & 125 & 1 & 2.6 & 0 & 0 & 3 & 0 \\
3 & 61 & 1 & 0 & 148 & 203 & 0 & 1 & 161 & 0 & 0.0 & 2 & 1 & 3 & 0 \\
4 & 62 & 0 & 0 & 138 & 294 & 1 & 1 & 106 & 0 & 1.9 & 1 & 3 & 2 & 0 \\
\bottomrule
\end{tabular}
\caption{Sample Data}
\label{tab:sample_data}
\end{table}

در ادامه برای آموزش مدل مبتنی بر
Bayse
مجموعه داده موجود با استفاده از کد زیر به دو قسمت آموزش و ارزیابی تقسیم می‌شود و پس از آن مدل به روی دیتاست مذکور تنظیم می‌شود.

\begin{LTR}
\begin{lstlisting}[language=Python, caption=best alpha]
# Split data into training and testing sets
X_train, X_test, y_train, y_test = train_test_split(X, y, test_size=0.2, random_state=53)

# Initialize and train the Gaussian Naive Bayes classifier
model = GaussianNB()X_train, y_train)

# Train model
model.fit(X_train, y_train)
\end{lstlisting}
\end{LTR}

نتایج بدست آمده از مدل فوق روی مجموعه داده ارزیابی به شرح زیر می‌باشد. با توجه به متریک 
Recall
می‌توانیم به این نتیجه برسیم که مدل در پیدا کردن نمونه‌های کلاس 0، نسبت به نمونه‌های کلاس 1 ضعیف‌تر عمل کرده است و با اینکه تعداد نمونه‌ها در هر دو کلاس یکسان بوده اما دقت مدل در پیدا کردن کلاس 0، 
10\%
ضعیف‌تر بوده اما از سوی دیگر اگر مدل درمورد یک نمونه از کلاس 0 پیشبینی کرده، احتمال اینکه این پیشبینی درست باشد بیشتر است و این مسئله از متریک 
Precision
قابل نتیجه می‌باشد. به صورت کلی دقت مدل در پیدا کردن و میزان درست بودن پیشبینی‌هایش از رابطه‌ای بین 
Precision و Recall 
نتیجه می‌شود که به آن 
f-1 Score
می‌گویند. 
f-1 Score 
برای هردو کلاس بالای 
80\% 
است و اختلاف کمی دارد که نشان می‌دهد مدل علاوه‌براینکه از دقت مناسبی برخوردار است، به کلاس خاصی متمایل نشده است. این نتایج در ماتریس درهم‌ریختگی موجود در 
\autoref{fig: Q4 cm}
قابل ملاحضه است.
\begin{LTR}
\begin{verbatim}
Classification Report:
               precision    recall  f1-score   support

     Class 0       0.86      0.77      0.81       102
     Class 1       0.80      0.87      0.83       103

    accuracy                           0.82       205
   macro avg       0.83      0.82      0.82       205
weighted avg       0.83      0.82      0.82       205
\end{verbatim}
\end{LTR}

\begin{figure}[H] 
	\centering
	\subfigure[]{
		\includegraphics[width=0.4\linewidth]{img/Q4/cm.png}
		}
	\subfigure[]{
		\includegraphics[width=0.4\linewidth]{img/Q4/cmn.png}
		}
	\caption{Confusion Matrix}
	\label{fig: Q4 cm}
\end{figure}

در کتابخانه
 Scikit-learn،
 اصطلاحات 
 micro
  و 
  macro
   به روش‌های مختلف میانگین‌گیری معیارها مانند
   Recall، Precision و  f-1 Score
   در مسائل طبقه‌بندی چاشاره دارند. روش micro متریک‌ها را به صورت کلی در شرایطی هر پیش‌بینی ارزش یکسانی دارد محاسبه می‌کند، به عبارت دیگر این روش تمام 
   true positives، true negative، false negatives و false positives 
   را برای محاسبه متریک موردنظر حساب می‌کند
   که می‌تواند تحت تأثیر 
   imbalance
    بودن کلاس‌ها قرار گیرد اما عملکرد کلی جامعی را ارائه می‌دهد. در مقابل، میانگین‌گیری
    macro
     معیارها را برای هر کلاس به طور مستقل محاسبه و سپس میانگین می‌گیرد، و به هر کلاس بدون توجه به فراوانی آن وزن مساوی می‌دهد، که آن را نسبت به
     imbalance
      کلاس‌ها کمتر حساس می‌کند. انتخاب بین میانگین‌گیری micro و macro بستگی به این دارد که آیا می‌خواهید بر عملکرد کلی تأکید کنید یا عملکرد متعادل در تمام کلاس‌ها مدنظر می‌باشد.

در انتها به ارزیابی مدل روی 5 نمونه از داده‌ها می‌پردازیم. با استفاده از کد زیر 5 نمونه به صورت تصادفی از مجموعه داده استخراج می‌شود و سپس مدل روی این داده پیشبینی انجام می‌دهد و در گام آخر پیشبینی به همراه کلاس واقعی نمایش داده می‌شود که این نتایج در 
\autoref{tab:Q4 t}
قابل مشاهده می‌باشد. این مدل تنها در یک مورد از 5 مورد اشتباه کرده است که کلاس را 1 تشخیص داده اما این درحالی است که کلاس واقعی 0 بوده است.


\begin{LTR}
\begin{verbatim}
test_data = data.sample(n=5, random_state=53)
pred = model.predict(test_data.drop("target", axis='columns'))
test_data['predictions'] = pred
test_data.head()
\end{verbatim}
\end{LTR}

\begin{table}[H]
\centering
\begin{tabular}{cccccccccccccccc}
\toprule
 & Age & Sex & CP & Trestbps & Chol & FBS & Restecg & Thalach & Exang & Oldpeak & Slope & CA & Thal & Target & Predictions \\
\midrule
420 & 57 & 0 & 0 & 128 & 303 & 0 & 0 & 159 & 0 & 0.0 & 2 & 1 & 2 & 1 & 1 \\
706 & 57 & 1 & 2 & 128 & 229 & 0 & 0 & 150 & 0 & 0.4 & 1 & 1 & 3 & 0 & 1 \\
673 & 54 & 1 & 2 & 120 & 258 & 0 & 0 & 147 & 0 & 0.4 & 1 & 0 & 3 & 1 & 1 \\
246 & 54 & 1 & 1 & 192 & 283 & 0 & 0 & 195 & 0 & 0.0 & 2 & 1 & 3 & 0 & 0 \\
187 & 56 & 1 & 0 & 125 & 249 & 1 & 0 & 144 & 1 & 1.2 & 1 & 1 & 2 & 0 & 0 \\
\bottomrule
\end{tabular}
\caption{Test Data}
\label{tab:Q4 t}
\end{table}



\bibliographystyle{plain}
\bibliography{references}
\end{document}

